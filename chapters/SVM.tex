% ======================================================================
% PHẦN 1: BÀI TOÁN GỐC (PRIMAL) - LÝ THUYẾT & CODE
% ======================================================================

\cheatbox{1. Bài toán Gốc: Toán học (Theory)}{\textStrech}{
\footnotesize
\textbf{1. Thiết lập:}
\begin{itemize}[leftmargin=*, nosep]
    \item $x, w \in \mathbb{R}^{M-1}$; Siêu phẳng: $w^T x + b = 0$.
\end{itemize}

\textbf{2. Dẫn dắt:}
Max Margin $\Leftrightarrow$ Min $\|w\|$ $\Leftrightarrow$ Min $\frac{1}{2}\|w\|^2$.

\textbf{3. Công thức Primal:}
\vspace{-0.2cm}
$$ \boxed{\begin{aligned}
& \min_{w,b} \frac{1}{2}\|w\|^2 \\
& \text{s.t. } t_n(w^T x_n + b) \geq 1, \forall n
\end{aligned}} $$
\vspace{-0.4cm}
}

\cheatbox{2. Bài toán Gốc: Implementation (CVXOPT)}{\textStrech}{
\footnotesize
Solver: $\min \frac{1}{2}u^T P u + q^T u$ s.t $Gu \leq h$.
\textbf{Biến:} $u = \mathbf{[w; b]}_{(M \times 1)}$. \textbf{Mapping:}
\vspace{-0.2cm}
{\scriptsize
$$ \boxed{\begin{aligned}
P &= \mathbf{\text{diag}(1, \dots, 1, 0)}_{(M \times M)} \\
q &= \mathbf{0}_{(M \times 1)} \\
G &= - \mathbf{[ \text{diag}(t) \cdot X, \quad t ]}_{(N \times M)} \\
h &= -\mathbf{1}_{(N \times 1)}
\end{aligned}} $$
}
\vspace{-0.4cm}
}

% ======================================================================
% PHẦN 2: CHUYỂN TIẾP (LAGRANGIAN & KKT)
% ======================================================================

\cheatbox{3. Lagrangian \& Điều kiện KKT}{\textStrech}{
\footnotesize
\textbf{1. Hàm Lagrangian:}
$$ \mathcal{L} = \frac{1}{2}\|w\|^2 + \sum_{n=1}^N \underbrace{\alpha_n}_{\geq 0} \underbrace{\{1 - t_n(w^T x_n + b)\}}_{\substack{\text{đúng: }\leq 0 \\ \text{sai: }>0}} $$
$$ = \frac{1}{2}\|w\|^2 + E_{\text{data}} \text{ (cost trên dữ liệu)} $$

\textbf{Tính chất:} Với $\alpha_n \geq 0$, $(w^*,b^*)$ tối ưu:
$$ \boxed{\mathcal{L}(w^*,b^*,\alpha) \leq \frac{1}{2}\|w^*\|^2} $$

\textbf{2. Tính đạo hàm:}
\vspace{-0.1cm}
\begin{align*}
\frac{\partial \mathcal{L}}{\partial w} &= w - \sum_{n=1}^N \alpha_n t_n x_n \\
\frac{\partial \mathcal{L}}{\partial b} &= -\sum_{n=1}^N \alpha_n t_n
\end{align*}

\textbf{3. Điều kiện dừng (KKT-1):}
\vspace{-0.1cm}
\begin{itemize}[leftmargin=*, nosep]
    \item $\nabla_w \mathcal{L} = 0 \Rightarrow \boxed{w = \sum_{n=1}^N \alpha_n t_n x_n}$ (1)
    \item $\nabla_b \mathcal{L} = 0 \Rightarrow \boxed{\sum_{n=1}^N \alpha_n t_n = 0}$ (2)
\end{itemize}

\textbf{4. Các điều kiện KKT khác:}
\begin{itemize}[leftmargin=*, nosep]
    \item \textbf{(KKT-2)} Ràng buộc gốc: $1 - t_n(w^T x_n + b) \leq 0$
    \item \textbf{(KKT-3)} Ràng buộc dual: $\alpha_n \geq 0, \; \forall n$
    \item \textbf{(KKT-4)} Điều kiện bù: $\alpha_n\{1 - t_n(w^T x_n + b)\} = 0$
\end{itemize}

\textbf{Ý nghĩa:} Thỏa KKT $\Rightarrow$ $(w,b,\alpha) = (w^*,b^*,\alpha^*)$\\
\textbf{5. Tiêu chuẩn Slater (cho SVM):}

Nếu phần trong của tập khả thi không rỗng thì:
\[
\exists (w,b)\ \text{s.t.}\ 
1 - t_i(w^T x_i + b) < 0,\ \forall i
\]

\[
\Longrightarrow \text{strong duality }(p^* = d^*)
\;\Leftrightarrow\;
\text{duality gap} = 0
\]

\vspace{-0.2cm}
}

% ======================================================================
% PHẦN 3: BÀI TOÁN ĐỐI NGẪU (DUAL) - LÝ THUYẾT & CODE
% ======================================================================

\cheatbox{4. Bài toán Dual: Biến đổi (Math)}{\textStrech}{
\footnotesize
Thay (1), (2) vào $\mathcal{L}$:

\textbf{A. Thay $w$ vào $\frac{1}{2}w^T w$:}
\vspace{-0.2cm}
{\scriptsize % Giảm cỡ chữ để không tràn
$$ \frac{1}{2} \left(\sum \alpha_i t_i x_i\right)^T \left(\sum \alpha_j t_j x_j\right) = \frac{1}{2} \sum_{i,j} \alpha_i \alpha_j t_i t_j (x_i^T x_j) \; (*) $$
}

\textbf{B. Thay $w$ vào $-\sum \alpha_n t_n w^T x_n$:}
\vspace{-0.2cm}
{\scriptsize
$$ - \sum_{n} \alpha_n t_n \underbrace{\left( \sum_{m} \alpha_m t_m x_m \right)^T}_{w^T} x_n = -2 \times (*) $$
}

\textbf{C. Kết quả ($A + B = -A$):}
\vspace{-0.2cm}
$$ \boxed{\begin{aligned}
\max_{\alpha} g(\alpha) = \quad & \sum_{n} \alpha_n - \frac{1}{2} \sum_{i,j} \alpha_i \alpha_j t_i t_j (x_i^T x_j) \\
\text{s.t.} \quad & \alpha_n \geq 0; \quad \sum \alpha_n t_n = 0   \text{(}g(\alpha)\text{ luôn là hàm lồi)}
\end{aligned}} $$
\vspace{-0.4cm}
}

\cheatbox{5. Bài toán Dual: Implementation (CVXOPT)}{\textStrech}{
\footnotesize
Solver: $\min \frac{1}{2}\alpha^T P \alpha + q^T \alpha$ s.t $G\alpha \leq h, A\alpha = b$.
\textbf{Biến:} $\alpha \in \mathbb{R}^N$. \textbf{Mapping:}
\vspace{-0.2cm}
{\scriptsize
$$ \boxed{\begin{aligned}
P &= \mathbf{K_{\text{Gram}}}_{(N \times N)} \quad (K_{ij} = t_i t_j x_i^T x_j) \\
q &= -\mathbf{1}_{(N \times 1)} \quad (\text{Max } \Sigma \to \text{Min } -\Sigma) \\
G &= -I_{(N \times N)} ; \ h = \mathbf{0}_{(N \times 1)} \\
A &= t^T ; \ b = 0
\end{aligned}} $$
}
\vspace{-0.4cm}
}

% ======================================================================
% PHẦN 4: MỞ RỘNG
% ======================================================================

\cheatbox{6. Soft Margin \& Kernel}{\textStrech}{
\footnotesize

\textbf{1. Soft Margin (ý tưởng):}
Cho phép một số điểm vi phạm margin bằng biến slack $\xi_n$,
đổi lại bị phạt trong hàm mục tiêu.

\vspace{0.05cm}
\textbf{2. Bài toán gốc (Primal – soft margin):}
\[
\min_{w,b,\xi}\;
\frac12\|w\|^2 + C\sum_{n=1}^N \xi_n
\]
\[
\text{s.t. }\;
t_n(w^T x_n + b) \ge 1 - \xi_n,\quad
\xi_n \ge 0
\]

\textbf{Ý nghĩa $C$:}
$C$ lớn $\Rightarrow$ phạt mạnh (ít sai, lề hẹp);
$C$ nhỏ $\Rightarrow$ cho sai nhiều (lề rộng).

\vspace{0.05cm}
\textbf{3. Dạng đối ngẫu (Dual):}
{\scriptsize
\setlength{\jot}{1.5pt}
\[
\mathcal{L}
= \frac12\|w\|^2
+ C\sum_n \xi_n
- \sum_n \alpha_n\{t_n(w^T x_n + b) - 1 + \xi_n\}
- \sum_n \mu_n \xi_n
\]
}

\[
\max_{\alpha}\;
\sum_{n=1}^N \alpha_n
-
\frac12
\sum_{i,j}
\alpha_i \alpha_j t_i t_j\, x_i^T x_j
\]
\[
\text{s.t. }\;
\boxed{0 \le \alpha_n \le C},\quad
\sum_{n=1}^N \alpha_n t_n = 0
\]

\textbf{Phân loại theo KKT:}
\begin{itemize}[leftmargin=*, nosep]
    \item $\alpha_n = 0$: điểm ngoài margin (không support)
    \item $0 < \alpha_n < C$: support vector trên margin
    \item $\alpha_n = C$: support vector vi phạm / phân loại sai
\end{itemize}

\vspace{0.05cm}
\textbf{4. Kernel Trick:}
Thay tích vô hướng:
\[
x_i^T x_j \;\longrightarrow\; k(x_i, x_j)
\]
\[
f(x)
= \sum_{n\in SV} \alpha_n t_n\, k(x_n,x) + b,
\qquad
y = \mathrm{sign}(f(x))
\]
}
